\chapter{Fejlesztői dokumentáció}
\label{ch:dev}

\section{Bevezetés}

Az alkalmazáshoz való technológiák kiválasztásakor fontos mind a felhasználó, mind a fejlesztő igényeit figyelmbe venni. Szerencsére, jelen esetben van megoldás, amely minden oldal számára a legtöbb kényelmet nyújtja, mégpedig a webalkalmazás. A felhasználó számára könnyű elérést, platformfüggetlenséget, és egy megszokott felületet hordoz magával, ami különösen fontos egy oktatási céllal rendelkező alkalmazásnál, hiszen még kevesebb akadályt helyez a felhasználó és a "tananyag" közé. Fejlesztői szempontból is kényelmes egy ilyen alkalmazást a böngészőre írni, hiszen a JavaScript ökoszisztémában könyvtárak és keretrendszerek tömkelege áll rendelkezésre, melyek segítségével gyorsan és hatékonyan lehet egy webalkalmazást fejleszteni.

\section{Adatforrás}

Az adatok a BKK\nomenclature{BKK}{Budapesti Közlekedési Központ} által szolgáltatott OpenData Portálon\cite{bkkopendata} nyilvánosan elérhető adatbázisból származnak. Az adatokat a BKK a GTFS\index{GTFS -- General Transit Feed Specification} (General Transit Feed Specification) formátumban teszik elérhetővé, ami egy Google-nél kifejlesztett\cite{gtfsabout} nyilvánosan elérhető specifikáció, mely egy szabványos formátumot definiál a tömegközlekedési adatok szolgáltatására.

\section{Felhasznált technológiák}

\subsection{Alkalmazás felépítése}

Az alkalmazás egy backendből és egy frontendből áll, REST API-n\nomenclature{REST}{REpresentational State Transfer, egy szoftverarchitektúra típus, ami megkötésekkel garantálja többek között az adatok gyorsítótárazhatóságát\cite{rest}} keresztül kommunikálnak egymással. A backend feladata a GTFS formátumban elérhető adatok adatbázisba való betöltése, valamit ezen adatok szolgáltatása a frontend számára. A frontend egy webalkalmazás, mely a felhasználói felületet biztosítja a felhasználók számára.

Fontos megemlíteni, hogy az útvonal tervezése és az algoritmusok futtatása a frontenden történik. Azért választottam ezt a megoldást, hogy az API-n átvitt adatok komplexitását minimalizáljam; mivel a frontendnek egyébként is szüksége van az összes információra az algoritmus belső állapotáról, így a számításokat a frontendre helyezve elég az adatbázis-lekérdezéseket és azok eredményét kommunikálni a kettő között.

\subsection{Verziókövetés}

Fejlesztéshez a \textit{git} használatával tartom számon a változásokat, így a fejlesztés során bármikor visszaállítható egy korábbi verzió, vagy összehasonlítható két verzió közötti különbség.

\subsection{Backend felépítése}

A backend egy \textit{Node.js}\footnote{A \textit{Node.js} egy platform, ami JavaScript kód szerveroldali futtatását teszi lehetővé} alapú alkalmazás, mely az \textit{Express.js} keretrendszert használja a REST API megvalósítására. Fontos tényező volt a környezet kiválasztásában, hogy az NPM\footnote{Az NPM egy csomagnyilvántartás JavaScript csomagoknak, saját állításuk szerint a világ legnagyobb csomagnyilvántartása\cite{nodeabout}}-en megtalálható \textit{node-gtfs}\cite{nodegtfs} csomag egyike volt a kevés elérhető könyvtáraknak\footnote{A GTFS adatok feldolgozására való könyvtárak listája megtalálható a \url{https://gtfs.org/resources/gtfs/} oldalon \textit(Letöltés dátuma: 2024.11.22.)}, amelyek képesek GTFS adatok adatbázisba való betöltésére és lekérdezésére. További előnye a \textit{Node.js} backend választásának, hogy a frontenddel azonos a fejlesztői környezet, így a fejlesztéshez nem kell új programokat telepíteni, és a frontend és a backend fejlesztése közötti váltást is egyszerűvé teszi.

Hogy a backend akár távoli szerveren is egyszerűen beindítható legyen a teljes tesztkörnyezet reprodukálása nélkül, az alkalmazást \textit{Dockerizáljuk}\index{Docker -- virtualizációs technológia, mely alkalmazások platformfüggetlen futtatását teszi lehetővé}; így egy \texttt{git clone [repo] \&\& docker compose up -d} paranccsal bárhol egyszerűen futtatható a backend (ahol a Docker és a git telepítve van).

A kód olvashatósága és karbantarthatósága érdekében a backend kódja TypeScript\footnote{A TypeScript egy nyelv, ami a JavaScriptre épül, de statikus típusokat is támogat\cite{typescript}} nyelven íródik, ami JavaScriptre fordul, így a Node.js is képes futtatni.

\subsection{Frontend felépítése}

A frontend egy \textit{React.js}\footnote{A \textit{React.js} egy JavaScript könyvtár, melyet a Facebook fejlesztett ki, és felhasználói felületek fejlesztésére szolgál} alapú webalkalmazás, mely a backendhez hasonlóan \textit{TypeScript} nyelven van írva. A \textit{React.js} egy komponens alapú keretrendszer, melynek segítségével a felhasználói felületet kisebb, önállóan működő komponensekre bonthatjuk, így a kód olvashatóbb és karbantarthatóbb lesz. Azért erre esett a választás Angular és Vue helyett, mert a React népszerűsége messze túlszárnyalja ezekét\cite{reactcomparison}, így a fejlesztők számára könnyen elérhetőek a segédanyagok és a közösség támogatása is.
