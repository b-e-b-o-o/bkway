\chapter{Bevezetés}
\label{ch:intro}

Az informatika - különösen egyetemi környezetben való - oktatásában az elméleti háttér kiemelkedő szerepet kap. Ez természetes, hiszen megfelelő elméleti alapok nélkül a gyakorlatban is csak korlátozottan lehet eredményeket elérni. Azonban sok tanuló számára a száraz elméleti anyag nehezen érthető, és gyakorlati alkalmazás hiányában gyakran érdektelennek tűnik. Az elvont elméletet nehéz lehet a gyakorlattal összekapcsolni, és sokaknak ez okozza a legnagyobb nehézséget az informatika tanulásában. Ez a probléma különösen szembetűnő az algoritmusok tanulásakor, hiszen ezek eredendően gyakorlatiasak; céljuk a program gyorsabb, vagy más szempontból eredményesebb működése. Ám ez a gyakorlatiasság sokszor elveszik az elméleti leírásokban, és a hallgatók számára nehezen érthetővé válik.

Ennek a programnak a célja, hogy segítséget nyújtson az útkereső algoritmusok megértésében az elmélet és a gyakorlat összekapcsolásával. Az alkalmazás egy webes felületen keresztül teszi lehetővé a felhasználók számára, hogy különböző algoritmusok működését vizsgálhassák lépésről lépésre. Erre célra mi sem jobb adatforrás, mint a budapesti tömegközlededés, amivel a magyar diákok jelentős része nap mint nap találkozik. Az alkalmazásban a felhasználók négy alapvető útkereső- és gráfbejáró-algoritmus működését hasonlíthatják össze: a szélességi keresést, a mélységi keresést, a Dijkstra-algoritmust és az A*-algoritmust.

\pagebreak
Az alkalmazás használata során a felhasználók kiválaszthatnak egy kiinduló és egy célállomást, majd az alkalmazás lépésről lépésre bemutatja az adott algoritmus működését a két állomás közötti útvonal megtalálásához. A grafikus felület segít az egyes algoritmusok előnyeinek és hátrányainak a megértésében, és akár az algoritmusok ismeretében kevésbé jártas felhasználók számára is egy képet ad azok működési elvéről.
